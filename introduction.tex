\chapter{Introduction}
\label{c:intro}

With the arrival of big-data era, similarity search for various kinds of datasets among distributed machines like mobile devices has become a common and important task.  In a distributed environment where a large amount of local machines are involved in computation and storage, our goal is to minimize the amount of transmission cost while identifying the nearest neighbors of a query. This paper aims to generalize the current state-of-the-art on distributed pattern matching from only suitable for ``time series''  datasets to be as general as possible for other types of data such as image, audio, or even video dataset too.


% Scenario
Consider the first scenario in which, through a program installed on mobile devices, a company about data mining could obtain some unusual and potentially profitable images and want to know quickly whether a similar image exsists in someone's cell phone or tabelet.  To do so, it is required to find the similar images among a large amount of distributed devices for the final answer.  Moreover, this company could construct a platform which allows users to search among the images of other cell phones as long as they also provide their images.  With this platform, each user could use their images to find the similar images from other cell phones.  On the other hand, beside of the images, this platform could also deal with other type of data like music.  

Since this paper is generalized from the work \cite{LeeWave} called LeeWave and our work \cite{MsWave} called MsWave, our model could also apply to those scenarios such as the abnormal detection and work on the time series datasets. Although our discussions in the following chapters would focus on the problem of handling a single query and finding the $k$ nearest neighbors among these distribued devices, we could deal with multiple queries and the $k$ farthest neighbors like MsWave only with a little modifications.

% Challenge
The proposed task is challenging in several aspects. First, the goal is to identify the \emph{exact} $k$ nearest neighbors with guarantee, instead of approximated $k$ nearest neighbors.  Second, we are targeting at handling endless number of queries (simliar to a search engine) and assume the data are stored distributed in many local devices (thousands or even millions).  Finally, different from most of the previous works to focus only on time-series data, we are not constraining ourselves to a specific type of data, rather emphasizes on a general-purpose algorithm.

% Weakness of LeeWave.
One of the state-of-the-art method, LeeWave,\cite{LeeWave} has been proposed to handle some of the above challenges.  Unfortunately, it is designed specific for time-series data and works poorly for other types of data.  The main reason is that the way LeeWave used to prune the candidates assumes the smoothness of data, which is not necessary true when the data are not time series.  Another problem is that the cost of the pruning procedure in LeeWave grows linearly with the number of total instances in these distrubited devices.  When the number of instances becomes large, so is the communication cost.

% Our model.

We propose a two-phase framework to solve these challenges.  On the basis of LeeWave, our proposed model also derives theoretical bounds for pruning candidates in the early stages.  To further improve the performance of pruning, we add a preprocessing step which directly optimizes these bounds for our data without the assumption its property.  After that, the optimized bounds are not only more powerful than those of LeeWave for time-series data, but also able to be applied on other types of data.  We also devise a new method to make the cost during the pruning procedure independent of the number of total instances, which could significantly reduce the communication cost as we are dealing with large-scale data.


% Contribution
We summarize our main contributions as follows:
\begin{enumerate}
	\item We propose a general communication-efficient framework to identify exact $k$NN instances given a query. The model can be applied to various types of datasets which are distributed in a large amount of devices.
	\item In the part of methodology, based on the ideas of upper bounds and lower bounds of similarity in \cite{LeeWave}, we derive new bounds with the help of the orthogonal transformation.  And these new bounds enhance the power of pruning is the main source to achieve the goal of cost saving in communication. 
	\item Since it would be expensive to find the threshold used in the pruning procedure if the number of total instances in these machines is large, we give a method that could make this cost be indenpendent of this number.
	\item We propose a method to estimate the communication cost which allows us to dynamically adjust how much information we need to send for the current query according to the history of the past queries.
	\item We conduct extensive experiments for various types of datasets to demonstrate the effectiveness of our model.
\end{enumerate}

This thesis is organized as follows: In the chapter 2, we discuss the related works and their limitations.  In the chapter 3, we describe the details of our framework. In the chapter 4, we provide the results of the experiments using images, audio, and time series datasets, and conclude our works in chapter 5.

%\bibliographystyle{unsrt}
%\bibliography{thesisbib}