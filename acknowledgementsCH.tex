\begin{acknowledgementsCH}

能夠完成這篇論文實在要感謝太多人了,由於篇幅有限,無法全部一一列出。
\\
首先要感謝的是在我台大碩士生活中主要指導我的林守德教授。不只是因為他在課業上專業的建議與啟發,在生活上也是常常關心勉勵我的一位好老師!能在他的帶領下做研究是我碩班生活中非常美好的經驗。
\\
再來要感謝的是與守德老師共同指導我的葉彌妍教授。由於我一進實驗室就被分配到由她跟守德教授一起帶領的Intel團隊,她也給了我非常多寶貴的建議。不只是這篇碩士論文,我們去年上ICDM的那篇論文都是從葉老師之前在VLDB的論文衍生出的。我的碩士班生活能夠同時有兩位如此優秀的教授共同指導實在是非常紮實又幸福的一件事。
\\
接著要感謝的是盧昱辰同學,我們是在同一個時間加入守德老師的實驗室。在研究上,我們一起加入了Intel團隊,並且兩人一起做題目。由於我們常常互相討論,他總是能夠迅速指出我的推論可能存在的瑕疵,並進一步解決。因此跟他討論是一件非常愉快的一件事,總是能感到自己有所收獲。
\\
還要感謝我的父母家人。沒有他們的支持鼓勵就絕對不會有今天的我。由於我們家的兄弟姊妹很多,他們在如此大的壓力下無怨無悔地拉拔我們長大成人,一直都讓我非常敬佩以及感激。
\\
然後要感謝的是我的女朋友蘇柔郡。雖然專業的領域不同,但她還是能夠在一些我忽略的盲點上給我建議。並且能夠體諒我忙碌的研究生日子,實在是我這段日子最大的精神寄託。
\\
要感謝的人實在是太多了,學校的教授,指導我過的學長姊,互相切磋的同學,以及同居的室友。從大家身上我都學到了很多,也讓我期許自己日後不但要能湧泉以報,還要能像他們一樣能夠造福後人。

\end{acknowledgementsCH}