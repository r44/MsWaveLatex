\chapter{Methodology}
\label{c:method}

HiHi Iam r44~. The organization of this thesis is as follows. In chapte~\ref{c:thm}, the theoretical background and definition of surface plasmon will be included~\cite{maier2007plasmonics}. Chapte~\ref{c:exp} contains description of experiment methods such as atomic force microscopy and scanning electron microscopy. 

%==========================================================================================
\section{Problem Setup}
\label{s:probsetup}
There are a query set $Q={\{q_1,q_2,...,q_T\}}\subset\mathbb{R}^D$ at the server $P$ and a dataset $X_i\subset\mathbb{R}^D$ on each local machine $M_i$.  For each coming query $q_t$, we want to find its $k_{th}$ nearest neighborhood among these distributed datasets while reducing the transmission cost between $P$ and each $M_i$.

%==========================================================================================
\section{Overall Framework}
\label{s:overall}
In this section, we describe the overall framework of our work.  Then, we will give the details about the framework in the following sections.

There are two main phases in our framework.  For each $X_i$, the first phase only needs to be done for once.  On the other hand, we need to run the second phase for each new query $q_t$.

The first phase is an preprocessing procedure for the second phase. Its goal is to improve the performance of pruning in the second phase.  We will prove in the Sec \ref{ss:derivation_of_the_bounds} that this pruning power is highly correlated to the distribution of the norm of the feature vectors. As a result, each $M_i$ would learn an othorogal matrix $W_i$ for its $X_i$ to fit our desired distribution and then send each $W_i$ back to $P$.  We can notice that this phase is only dependent on $X_i$ and independent of $q_t$.  Therefore, we only need to do the first phase for once.  we give the details about how to learn $W_i$, how to send it back to $P$ in the Sec \ref{s:orthogonal}.

The second phase is the main procedure of our framework.  Note that $P$ have already got $W_i$ for each $X_i$ in the beginning of the second phase. For each coming query $q_t$, we iteratively prune some candidates which are impossible to be the $k$NN of $q_t$ to reduce the search space until there are only $k$ candidates left.

To prune candidates iteratively, we divide the second phase into several rounds.  For each round $j$, we use a $Select$ function $S_i(q_t,j;\theta_t)$ to generate the values for trasmitting from $P$ to $M_i$, where $S_i$ is the importance-selecting function of $M_i$ and $\theta_t$ is its parameters for $q_t$.  (We put the details of $S_i$ at the Sec \ref{s:importance_selecting_function}.)  By these values, each $M_i$ could calculate the bounds between each candidate $x_l$ and $q_t$.  With these bounds, $P$ would be able to determine which candidates are definitely not our answer and then disregards them in the following rounds.  By these pruning, we could achieve the goal of saving transmission cost from avoiding to consider the unnecessary candidates.

Note that we could use the square of the Euclidean distance instead of the origin Euclidean distance to find $k$NN as it is non-negative.  So we will use the former one in our framework.

%==========================================================================================
\section{Orthogonal Transformation}
\label{s:orthogonal}
Since the In this section, we describe the overview of this thesis.

%==========================================================================================
\subsection{Definition of Ortohogonal Transformation}
\label{ss:ortho_def}
\newtheorem{Orthogonal}{\bf Definition}
\begin{Orthogonal}
A matrix $W \in\mathbb{R}^{D\times D}$ is orthogonal if whose columns and rows are orthogonal vectors, i.e.
\[
W^{T}W=WW^{T}=I
\]
where $I$ is the identity matrix.
\end{Orthogonal}

%==========================================================================================
\subsection{Property of Orthogonal Transformation}
\label{ss:ortho_prop}
\newtheorem{ProOfOrthogonal}{\bf Property}
\begin{ProOfOrthogonal}
Let $x, y\in\mathbb{R}^{D\times 1}$, and $W\in\mathbb{R}^{D\times D}$ be an orthogonal matrix. Then,
\[
Dist(x,y)=\sum^D_{d=1}{(x[d]-y[d])^2} \\
=\sum^D_{d=1}{(W[d,:]x-W[d,:]y)^2}=Dist(Wx,Wy)
\]
where $W[d,:]$ is the $d_{th}$ row of $W$.
\end{ProOfOrthogonal}

We will use this important property in the Sec \ref{sub:reduce_the_norm_with_orthogonal_transformation}.

%==========================================================================================
\section{Enhance the Bounds by the Orthogonal Transformation}
\label{s:ortho_bounds}
In the Sec \ref{s:overall}, we mentioned that the first phase is an auxiliary step for the second phase.  After the introduction of the orthogonal transformation, we introduce this powerful tool into the first phase in our framework.

\subsection{The Goal of the First Phase} % (fold)
\label{ss:the_goal_of_the_first_phase}

Our goal in the first phase is to reduce the ranges of the bounds used in the second phase.  Since we will use a threshold to prune the impossible candidates according to their bounds in the The pruning procedure, the ranges of the bounds would be one of the most influential factor of the pruning power.


% TODO: put some graph to compare the long ranges of bounds with the short range.

Suppose that we want to prune all candidates whose lower bounds are upper than a threshold $thr$.  In these figures, we could see that when the ranges of their bounds is short, more candidates would be pruned than those with long ranges of the bounds.  In other words, the shorter the range of the bound, the higher chance this candidate would be pruned if it is not our final answer of $k$NN.

% subsection the_goal_of_the_first_phase (end)

\subsection{Definition of the Bounds} % (fold)
\label{ss:definition_of_the_bounds}

First, we need to define the bounds for the pruning.  Recall that given a query $q_t$, our goal is to find its $k$NN in these distributed datasets $X_i$.  Intuitively, we need to calculate the square of the Euclidean distance $Dist(q_t,x), \forall x\in \cup_i X_i$.  However, to cacluate $Dist(q_t,x)$, we need to send the whole $q_t$ to the local machines or send the whole $x$ to $P$, which causes a huge transmission cost.  Therefore, instead of the exact value of Euclidean distances, our propsed framework uses bounds to find the $k$NN .

\newtheorem{Bounds}{\bf Definition}
\begin{Bounds}
$\forall x,y \in \mathbb{R}^D$, a lower bound $LB(x,y)$ and a upper bound $UB(x,y)$ must satisfy the following inequation:
\[
LB(x,y)\leq Dist(x,y) \leq UB(x,y)
\]
\end{Bounds}

\subsection{Relation Between the Norms and the Bounds} % (fold)
\label{sub:relation_between_the_norms_and_the_bounds}

To acheive the goal of reducing the length of the ranges of the bounds, we could look into the derivation of the bounds. Suppose there are two vectors $x,y\in \mathbb{R}^D$, but we could only observe the first $s$ dimensions of $x$ and $\sum^D_{d=s+1}{x[d]^2}$, which is the square of two norm of the unobserved part $x[s+1:D]$.  In the Sec \ref{ss:derivation_of_the_bounds}, we give the bounds as 
{
\begin{eqnarray*}
\lefteqn{LB(x,y) = \sum^s_{d=1}{(x[d]-y[d])^2}.} \label{eq:single-LB-tmp} \\
\lefteqn{UB(x,y) = \sum^s_{d=1}{(x[d]-y[d])^2}} \notag \\
& + & \sum^D_{d=s+1}{x[d]^2}+\sum^D_{d=s+1}{y[d]^2} \notag \\
& + & 2 \times \sqrt{\sum^D_{d=s+1}{x[d]^2}\times \sum^D_{d=s+1}{y[d]^2}.}\label{eq:single-UB-tmp}
\end{eqnarray*}
}

Therefore, we could get the length of the range by the substraction.
\[
Len = UB(x,y) - LB(x,y) = \sum^D_{d=s+1}{x[d]^2}+\sum^D_{d=s+1}{y[d]^2} \notag \\
 +  2 \times \sqrt{\sum^D_{d=s+1}{x[d]^2}\times \sum^D_{d=s+1}{y[d]^2}.}\label{eq:len-range}
\]

Since the term $\sum^D_{d=s+1}{x[d]^2}$ is given, all we could do is to reduce the length with the help of the $\sum^D_{d=s+1}{y[d]^2}$ term.  If we could reduce $\sum^D_{d=s+1}{y[d]^2}$, the term $\sum^D_{d=s+1}{x[d]^2}\times \sum^D_{d=s+1}{y[d]^2}$ would also decrease and then make $Len$ smaller.  Therefore, our goal now becomes to make the term $\sum^D_{d=s+1}{y[d]^2}$ as small as possible, which is the square norm of the vector $y[s+1:D]$.

Note that the lower (upper) bounds is non-decreasing (non-increasing) as $s$ becomes larger and $LB(x,y)=UB(x,y)$ when $s=D$.  This means that $LB$ and $UB$ would be exactly equal to $\sum^D_{d=1}{(x[d]-y[d])^2}$ eventually.

To reduce the length of the ranges $Len$ for each $s$, we hope to make $\sum^D_{d=s+1}{y[d]^2}$ as small as possible.  However, since the feature vector $y$ is given from datasets, the value of $\sum^D_{d=s+1}{y[d]^2}$ is already  determined when $s$ is given.  As a result, we introduce the orthogonal transformation to achieve this goal.

% subsection relation_between_the_norms_and_the_bounds (end)


\subsection{Equivalent Bounds After Transformation} % (fold)
\label{sub:equivalent_bounds_after_transformation}

From the Sec \ref{ss:ortho_prop}, we know the distance of two vectors won't be changed after an orthogonal transformation.  Now we use this property to achieve our goal to reduce the length of the ranges $Len$ given $s$.

Given an orthogonal transformation $W\in\mathbb{R}^{D\times D}$, we have $Dist(x,y)=Dist(Wx,Wy)$.  This means that the bounds we derivated before could also be the bounds for $Dist(Wx,Wy)$.  That is,
\[
LB(x,y)\leq Dist(x,y)=Dist(\hat{x},\hat{y}) \leq UB(x,y)
\]
where $\hat{x}=Wx$.

This also means that we could use the same way to derivate the lower bounds and upper bounds for $Dist(Wx,Wy)$ and these bounds are also the bounds for $Dist(x,y)$.  That is,
\[
LB(\hat{x},\hat{y})\leq Dist(x,y)=Dist(\hat{x},\hat{y}) \leq UB(\hat{x},\hat{y})
\]

So, we could use the bounds $LB(\hat{x},\hat{y}),UB(\hat{x},\hat{y})$ for $Dist(x,y$ and the length of range we want to reduce becomes
\[
\hat{Len}(W) = UB(\hat{x},\hat{y}) - LB(\hat{x},\hat{y}) = \sum^D_{d=s+1}{\hat{x}[d]^2}+\sum^D_{d=s+1}{\hat{y}[d]^2} \notag \\
 +  2 \times \sqrt{\sum^D_{d=s+1}{\hat{x}[d]^2}\times \sum^D_{d=s+1}{\hat{y}[d]^2}.}\label{eq:len-range-hat}
\]
which becomes a function of $W$.

As a result, instead of trying to reduce $\sum^D_{d=s+1}{y[d]^2}$ which is impossible as we mentioned in the Sec \ref{sub:relation_between_the_norms_and_the_bounds}, our goal becomes to reduce $\sum^D_{d=s+1}{\hat{y}[d]^2}$ with the help of $W$.

% subsection equivalent_bounds_after_transformation (end)

\subsection{Reduce the Norm with Orthogonal Transformation} % (fold)
\label{ss:reduce_the_norm_with_orthogonal_transformation}

For $y\in \mathbb{R}^{D\times 1}$ and $W\in\mathbb{R}^{D\times D}$, given $s$, we want to reduce $\sum^D_{d=s+1}{\hat{y}[d]^2}$ as much as possible.  However, since $s$ is unknown while deciding $W$ in the first phase of our framework, we have to handle all possible values which $s$ could be.  Moreover, since $\sum^D_{d=1}{\hat{y}[d]^2}$ is equal to $\sum^D_{d=s+1}{y[d]^2}$, which is independent with $W$, if the $\sum^D_{d=s+1}{\hat{y}[d]^2}$ decreases with some $W$, the term $\sum^s_{d=1}{\hat{y}[d]^2}$ must increase.  Here, we use a more general strategy to deal with these problems.

We could look the term $\sum^D_{d=s+1}{\hat{y}[d]^2}$ from a different angle.  Actually, this term is the square norm of the latter part of the vector $\hat{y}$. Therefore, although $\sum^D_{d=1}{\hat{y}[d]^2}$ is a constant for $W$, we could reduce the square norm of the latter part by increasing the forward part of it.  In other words, we move the norm of the latter part of $y$ to its forward part.  To accomplish it, we design an objective function and then optimize this function to find our ideal $W$.

\begin{equation}\label{objective}
	f(W;y)=\sum^D_{d=1}{w_d\times\hat{y}[d]^2}=\sum^D_{d=1}{w_d\times(W[d,:]y[d])^2}
\end{equation}
where $w_d=d,  \forall d=1:D$.

Because $w_d$ would give the larger penalty as $d$ increases, the elements in the latter part of $\hat{y}$ would be forced to become small while minimizing this objective function.  This is exactly our goal to reduce $\sum^D_{d=s+1}{\hat{y}[d]^2}$.  Therefore, our question becomes how to optimize this objective function with the constraints that $W$ must be an orthogonal matrix.


% subsection reduce_the_norm_with_orthogonal_transformation (end)

\subsection{Optimize with Orthogonal Constraints} % (fold)
\label{ss:optimize_with_orthogonal_constraints}

Finally, we could introduce this concept of reducing the norms into our framework.  In the first phase, we solve the following optimization problem to learn an orthogonal matrix $W_i$ for each machine $Mi$.

\begin{equation}
\begin{aligned}
& \underset{W}{\text{minimize}}
& & F_i(W) \\
& \text{subject to}
& & W^{T}W=WW^{T}=I
%& & W \leq b_i, \; i = 1, \ldots, m.
\end{aligned}
\end{equation}

where 
\begin{equation}\label{objective-F}
	F_i(W)=\sum_{x \in X_i}{f(W;x)}\\
	=\sum_{x \in X_i}{ \sum^D_{d=1} {d\times(W[d,:]x[d])^2} }
\end{equation}.

We solve this optimization problem with the help of the package ~~. 
%http://optman.blogs.rice.edu/

After we get the optimal $W^*_i$ for each $M_i$, we send these matrices back to the server $P$.  Since the learning of $W^*_i$ is independent with the queries in the future, we only have to go through the procedure of learning $W^*_i$ for once if $X_i$ doesn't change too much.

% subsection optimize_with_orthogonal_constraints (end)

\subsection{Reduce the Cost of Sending Matrices} % (fold)
\label{ss:reduce_the_cost_of_sending_matrices}


% subsection reduce_the_cost_of_sending_matrices (end)


%==========================================================================================
\section{Prune by the Bounds}
\label{s:prune}
Now we start to discuss the second phase of our framework.  In this section, we talk about how to prune the candidates if we already have bounds.  Note that this mechanism is the most crucial part to achieve our goal to save the transmission cost.

% subsection definition_of_bounds (end)

\subsection{Prune the Candidates with the Bounds} % (fold)
\label{ss:prune_the_candidates_with_the_bounds}

For the query $q_t$, if we already know $LB(q_t,x)$ and $UB(q_t,x)$ $\forall x\in \cup_i X_i$, we could use the $k_{th}$ smallest upper bounds and directly prune those $x$ whose lower bounds are higher than this value $thr$. I.e., we want to prune
\[
\{x |LB(q_t,x)>thr, \forall x \in \cup_i X_i\}
\]
where $thr$ is the $k_{th}$ largest $UB(q_t,x)$ $\forall x\in \cup_i X_i$.

Here is an example of the pruning procedure.

TODO: Draw the figure about pruning.

We will talk about the details to generate these bounds and find the threshold in the Sec~~haha.

% subsection prune_the_candidates (end)

\subsection{Derivation of the Bounds} % (fold)
\label{ss:derivation_of_the_bounds}

Suppose there are two vectors $x,y\in \mathbb{R}^D$, we know the square of their Euclidean distance is 
\begin{equation}
	Dist(x,y)=\sum^D_{d=1}{(x[d]-y[d])^2}
\end{equation}

However, if we could only observe the first $s$ dimensions of $x$, we could decompose their distance as 
\begin{equation}\label{eq:Eu_decomp}
	Dist(x,y)=\sum^D_{d=1}{(x[d]-y[d])^2}\\
				=\sum^s_{d=1}{(x[d]-y[d])^2} + \sum^D_{d=s+1}{(x[d]-y[d])^2.}	
\end{equation}

Since the first component of \eqref{eq:Eu_decomp} is already known, all we need to do is to deal with the second term. Therefore, we further expand the second term as following:
\[
\sum^D_{d=s+1}{(x[d]-y[d])^2}=\sum^D_{d=s+1}{x[d]^2}+\sum^D_{d=s+1}{y[d]^2}-\sum^D_{d=s+1}{2\times x[d]\times y[d].}
\]

By this analysis, we find the final term is the inner product between two partial vector $x[s+1:D]$ and $y[s+1:D]$, which could be approximated by Cauchy–Schwarz inequality
\begin{equation}\label{eq:Cauchy}
	\sum^D_{d=s+1}{x[d]\times y[d]} \leq \sqrt{\sum^D_{d=s+1}{x[d]^2}\times \sum^D_{d=s+1}{y[d]^2}.}
\end{equation}

After combing with \eqref{eq:Eu_decomp} and \eqref{eq:Cauchy}, we derive the bounds as 
{
\begin{eqnarray}
\lefteqn{LB(x,y) = \sum^s_{d=1}{(x[d]-y[d])^2}.} \label{eq:single-LB} \\
\lefteqn{UB(x,y) = \sum^s_{d=1}{(x[d]-y[d])^2}} \notag \\
& + & \sum^D_{d=s+1}{x[d]^2}+\sum^D_{d=s+1}{y[d]^2} \notag \\
& + & 2 \times \sqrt{\sum^D_{d=s+1}{x[d]^2}\times \sum^D_{d=s+1}{y[d]^2}.}\label{eq:single-UB}
\end{eqnarray}
}
We could notice that the calculation of the bounds only needs the first $s$ dimensions of $x$ and $\sum^D_{d=s+1}{x[d]^2}$.  Therefore, we only need one more number to get the bounds for the unobserved part $x[s+1:D]$.

% subsection derivation_of_bounds (end)

\subsection{Calculation of the Bounds} % (fold)
\label{sub:calculation_the_bounds}
After derivation of the bounds, we describe the procedure of cacluating them in our framework.  

For the query $q_t$, at the first round (i.e. $j=1$), $P$ sends the first $s_1$ dimensions of $q_t$ and $\sum^D_{d=s_1+1}{q_t[d]}^2$ to each $M_i$ .  With these values, each $M_i$ would able to calculate the lower bounds $LB(q_t,x)$ and upper bounds $UB(q_t,x)$ for each $x\in X_i$.  Then, after $P$ getting the $k_{th}$ smallest upper bounds as $thr$, we could run the pruning procedure.

In each following round (i.e. $j>1$), $P$ sends the next $s_j$ dimensions of $q_t$ to each $M_i$ whose instances were not pruned completely.  These $M_i$ will update their bounds as following
{
\begin{eqnarray}
\lefteqn{LB_j(x,y) = LB_{j-1}(x,y)+\sum^{p_j}_{d=p_{j-1}+1}{(x[d]-y[d])^2}.} \label{eq:single-LB} \\
\lefteqn{UB_j(x,y) = LB_{j}(x,y)} \notag \\
& + & \sum^D_{d=p_j+1}{x[d]^2}+\sum^D_{d=p_j+1}{y[d]^2} \notag \\
& + & 2 \times \sqrt{\sum^D_{d=s+1}{x[d]^2}\times \sum^D_{d=s+1}{y[d]^2}.}\label{eq:single-UB}
\end{eqnarray}
}
where $p_j=\sum^j_{i=1}{s_i}$, $LB_j$ and $UB_j$ indicate the lower bounds and upper bounds at the round $j$ repectively.

We call those $p_j$ as pivots, which mean that each machine would observe the first $p_i$ elements of $q_t$ at the round $j$.  

% subsection calculation_the_bounds (end)


\subsection{Find the Threshold in Distributed Machines} % (fold)
\label{ss:find_the_threshold_in_distributed_machines}
The question now is to find the threshold $thr$ for pruning.  Since 
TODO: Put this section here or after derived the bounds.

% subsection find_the_threshold_in_distributed_machines (end)
%==========================================================================================

\section{Decide the Pivots} % (fold)
\label{s:decide_the_pivots}

The remaining problem is to decide how many dimensions (i.e. $s_j$) of $q_t$ we have to send from the server $P$ in each round $j$.  If we send too few dimensions, the bounds would be too loose to prune any candidates and we will spend much unnecessary cost in finding the thresholds.  On the other hand, if we send too many dimensions, although it could allow us to prune many candidates at once, it would send too many dimensions to some candidates which could be pruned by much fewer dimensions.  That would also lead to the waste the transmission cost.  Therefore, in this section, we propose a simple but effective method to decide how many dimensions of $q_t$ we should send from $P$ in each round.

\subsection{Estimate the Transmission Cost} % (fold)
\label{sub:estimate_the_transmission_cost}

From the discussion above, we could notice that the decision of pivots is highly dependent with 

% subsection estimate_the_transmission_cost (end)


\section{Coordinate Descent to Decide the Pivots} % (fold)
\label{sub:coordinate_descent_to_decide_the_pivots}

% subsection coordinate_descent_to_find_the_pivots (end)

% subsection decide_the_pivots (end)

%==========================================================================================
\section{Importance-Selecting Function and Overall Framework} % (fold)
\label{s:importance_selecting_function_and_overall_framework}
At each round $j$ for $q_t$, we need to decide what to send from $P$ to each $M_i$ and then calculate the bounds at $M_i$. 

In this section, we describe the overview of our framework.

% subsection importance_selecting_function (end)

%==========================================================================================


%\bibliographystyle{unsrt}
%\bibliography{thesisbib}