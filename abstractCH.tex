\begin{abstractCH}

隨著移動式裝置與網路的普及,在分散式的機器環境中進行相似性搜索,像是圖片搜尋,或是以哼聲來搜尋,變成一個日益重要的問題。在一個分散式的環境中,傳輸成本通常被認為是比計算量的成本來得重要的,由於一般而言前者會花費更多的資源。因此,這篇論文提出一個方法讓我們能夠利用正交轉換在大量的分散式機器環境下有效率地尋找確切的$k$個最近鄰居。建立於我們之前的論文上,我們推導出新的在歐式距離上的界,這些界使我們能只憑著部分的指令資訊在早期即判斷出不可能會是答案的樣本並刪除之。更進一步,我們提出了三個演算法能進一步地壓低找到答案所需的傳輸量。實驗結果顯示,在資料量到達百萬的程度以上時,我們的方法能夠在傳輸量的節省上顯著地勝過其他方法。
\\
\\
  關鍵字:確切相似性搜索、剪枝、傳輸量節省、正交轉換、最優化問題

\end{abstractCH}
